%\documentclass{acm_proc_article-sp}
\documentclass{sig-alternate}
\usepackage{paralist}
\usepackage{url}
\usepackage{amsmath,amssymb}
\usepackage{listings}
\usepackage{tikz}
\usepackage{color}
\usepackage{xspace}
\usetikzlibrary{trees}


\newcommand\ednote[1]{\typeout{There is still an editor's note!!!}%
  \footnote{EDNOTE: #1}}
\newcommand\edbf[1]{\typeout{There is still an editor's note!!!}%
  \textbf{EDNOTE: #1}}

\def\collapse#1{\textcolor{blue}{\ensuremath{\mathord{\blacktriangleleft}
\mathord{#1}
\mathord{\blacktriangleright}}}}
\newcommand{\latex}{\LaTeX\xspace}
\newcommand{\sre}{SRE\xspace}

% \DeclareUnicodeCharacter{B1}{\pm}
%% LANGUAGE      Markup definitions
\def\lstxml{
  \lstset{language=XML,
    basicstyle=\scriptsize,
    keywordstyle=\bfseries\ttfamily,
    identifierstyle=\ttfamily, 
    % commentstyle=\color{Brown},
    stringstyle=\ttfamily,
    showstringspaces=false,
    columns=[l]flexible, %% , basewidth={0.5em,0.4em}
    escapeinside={(@*}{*@)},
    deletekeywords={type,id}, % does not work!
    otherkeywords={encoding,
      mrow,math,mfrac,mi,msqrt,mo,mn,span,nobr,img,msup,maction,mtext}
  }
}
\def\lsthtml{
\lstset{language=html,
    basicstyle=\scriptsize,
    keywordstyle=\bfseries\ttfamily,
    showstringspaces=false,
    % otherkeywords={mfenced, open, close, separators, mrow, mi, mo, mn, math, 
    %   msup, role, parent, children, added}
}
}

\def\latex{\LaTeX}


\begin{document}
% \CopyrightYear{2016} 
% \setcopyright{acmcopyright}
% \conferenceinfo{Assets,}{}
% \isbn{}}\acmPrice{\$15.00}
% \doi{}


\title{Supporting Visual Impaired Learners in Editing Mathematics}
  

\numberofauthors{3}
\author{
  \alignauthor {T.V. Raman}\\
  \affaddr{Google, Inc.}\\
  \affaddr{Mountain View, CA}\\
  \email{\normalsize raman@google.com}
  %
  \alignauthor{Volker Sorge}\\
  \affaddr{Progressive Accessibility Solutions, Ltd.}\\
  \affaddr{University of Birmingham, UK}\\
  \email{\normalsize V.Sorge@progressiveaccess.com}\\
  %% Thanks does not seem to work.
}

\maketitle

\begin{abstract}
  
  While mathematical expressions can be expressed in dedicated Braille formats,
  these become quickly unwieldy for more advanced mathematics and moreover it
  becomes impossible for students to communicate them to their peers or
  professors. Consequently they need to start using a linear format that is
  commonly understand. And indeed on transitioning from high school to
  University students often have to learn {\LaTeX} to have a means to read
  advanced material as well as to be able to communicate with their peers and
  teachers.

\end{abstract}

\keywords{STEM Accessibility, Mathematics, MathJax}


\enlargethispage{10pt}
\section{Introduction}

\begin{enumerate}
\item Enable students to write and explore the math content they write.
\item Enable student to rearrange an expression and hear the effect.
\item Enable student to grab math content, read it, browse it and possibily
  manipulate it.
\end{enumerate}

MathJax, a rendering library for Mathematics on the web, now offers an
accessibility solution

The main purpose of the work is therefore to support 

\section{Background}
\label{sec:background}

We first give a brief introduction to the systems that are involved in our
project.

\subsection{Emacs Speak}
\label{sec:emacs-speak}

\subsection{Speech Rule Engine}
\label{sec:sre}

Speech Rule Engine (\sre) is a Javascript library that allows aural rendering of
XML expressions given sets of speech rules that specify how speech strings are
composed recursively via from aurally rendered subexpressions using an extended
xpath syntax.  \sre has originally been implemented in the context of Google's
ChromeVox screen reader~\cite{chromevox}. It has since been turned into a
standalone, open source Javascript library, that can be installed server side
for NodeJS applications as well as used in client side web applications.





\subsection{MathJax Node}
\label{sec:mathjax-node}

MathJax is a JavaScript library that ensures flawless rendering of formulas
across all platforms and browsers. It was originally built as a polyfill
solution, that is, a temporary software solution until a new web standard is
implemented in all browsers, for rendering mathematics until the web standard
for mathematics, MathML, would be sufficiently supported.  But ten years later
and despite MathML being part of the official HTML5 specification, the majority
of browsers still do not support rendering of MathML natively. Hence MathJax has
become the quasi standard for displaying Mathematics in web pages.
The advantage of MathJax is that it can take the three most common mathematical
authoring formats ---- {\latex}, ASCIIMath\cite{asciimath}, and MathML --- as
input and generate high quality rendered output either as HTML or as SVG.

Although MathJax is built as a client side solution, to be included into
webpages by content authors, there is a special NodeJS package release of
MathJax, which allows server side rendering of mathematical expressions from
either of its three input formats. The output is either SVG or bitmap image
formats (PNG) for legacy systems. Moreover, it allows to exploit MathJax's
ability to convert {\latex} into standardised MathML. 

MathJax node can also use \sre as a dependency to produce speech strings for
mathematical expression for inclusion into the rendered output, either as alt
texts for images, descriptions for SVG or via dedicated attributes into MathML.
Furthermore, \sre has recently been integrated into MathJax in the browser,
exploiting SRE's walker features in an effort to turn MathJax into a universal
rendering engine, that can also provide assistive technology features such as
aural rendering, subexpression exploration and synchronised
highlighting\cite{w4a} on top of the visual rendering.

\section{Technical Realisation}
\label{sec:technical-realisation}

\section{TTS Rendering of Mathematics}
\label{sec:rendering}


\section{Error Handling and Feedback}
\label{sec:error-handling}

One major difficulty presented to students new to {\LaTeX} is learning the intricacy

\section{Advanced Features}
\label{sec:error-handling}

\section{User Interface and Experience}
\label{sec:ui}

\section{Conclusions}
\label{sec:conc}

\bibliographystyle{plain}
\bibliography{assets16}

\end{document}


%%% Local Variables:
%%% mode: latex
%%% TeX-master: t
%%% End:
