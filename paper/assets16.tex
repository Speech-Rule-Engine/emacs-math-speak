%\documentclass{acm_proc_article-sp}
\documentclass{sig-alternate}
\usepackage{paralist}
\usepackage{url}
\usepackage{amsmath,amssymb}
\usepackage{listings}
\usepackage{tikz}
\usepackage{color}
\usepackage{xspace}
\usetikzlibrary{trees}


\newcommand\ednote[1]{\typeout{There is still an editor's note!!!}%
  \footnote{EDNOTE: #1}}
\newcommand\edbf[1]{\typeout{There is still an editor's note!!!}%
  \textbf{EDNOTE: #1}}

\def\collapse#1{\textcolor{blue}{\ensuremath{\mathord{\blacktriangleleft}
\mathord{#1}
\mathord{\blacktriangleright}}}}
\newcommand{\latex}{\LaTeX\xspace}
\newcommand{\sre}{SRE\xspace}

% \DeclareUnicodeCharacter{B1}{\pm}
%% LANGUAGE      Markup definitions
\def\lstxml{
  \lstset{language=XML,
    basicstyle=\scriptsize,
    keywordstyle=\bfseries\ttfamily,
    identifierstyle=\ttfamily, 
    % commentstyle=\color{Brown},
    stringstyle=\ttfamily,
    showstringspaces=false,
    columns=[l]flexible, %% , basewidth={0.5em,0.4em}
    escapeinside={(@*}{*@)},
    deletekeywords={type,id}, % does not work!
    otherkeywords={encoding,
      mrow,math,mfrac,mi,msqrt,mo,mn,span,nobr,img,msup,maction,mtext}
  }
}
\def\lsthtml{
\lstset{language=html,
    basicstyle=\scriptsize,
    keywordstyle=\bfseries\ttfamily,
    showstringspaces=false,
    % otherkeywords={mfenced, open, close, separators, mrow, mi, mo, mn, math, 
    %   msup, role, parent, children, added}
}
}

\def\latex{\LaTeX}


\begin{document}
% \CopyrightYear{2016} 
% \setcopyright{acmcopyright}
% \conferenceinfo{Assets,}{}
% \isbn{}}\acmPrice{\$15.00}
% \doi{}


\title{Supporting Visual Impaired Learners in Editing Mathematics}
%\subtitle{Extended Abstract}
  

\numberofauthors{3}
\author{
  \alignauthor {T.V. Raman}\\
  \affaddr{Google, Inc.}\\
  \affaddr{Mountain View, CA}\\
  \email{\normalsize raman@google.com}
  %
  \alignauthor{Volker Sorge}\\
  \affaddr{Progressive Accessibility Solutions, Ltd.}\\
  \affaddr{University of Birmingham, UK}\\
  \email{\normalsize V.Sorge@progressiveaccess.com}\\
  %% Thanks does not seem to work.
}

\maketitle

\begin{abstract}
  We present an extension to the Emacsspeak audio desktop that provides support
  for editing mathematics in {\LaTeX}. It integrates MathJax and the Speech Rule
  Engine to support users in reading {\LaTeX} sources, manipulating mathematical
  formulas and verifying correctness of the rendered output.
\end{abstract}

\keywords{STEM Accessibility, Mathematics, MathJax}


\enlargethispage{10pt}
\section{Introduction}


Mathematics is still the single most significant hurdle for inclusive education
for visually impaired students in the STEM subjects. In particular, in secondary
education increasing complexity of mathematical formulas makes it harder to
communicate content equally effectively to both sighted and blind students.
While mathematical formulas can be expressed in dedicated Braille formats, these
become quickly unwieldy for more advanced mathematics and moreover it becomes
impossible for students to communicate them to their peers or
professors. Consequently they need to start using a linear format that is
commonly understood. And indeed on transitioning from high school to University
students often have to learn {\LaTeX} to have a means to read and communicate
advanced material.

While there are some specialist editors that support editing mathematics for the
blind (e.g.~\cite{Pearson}) there is very little support for reading or
authoring {\LaTeX} documents directly.  In particular, it is difficult for
students to write mathematical content in {\LaTeX} while checking that the
rendered output indeed corresponds to the mathematical formula they intended to
write. We aim to support this task by extending the Emacsspeak audio
desktop~\cite{Emacsspeak} with a mathematics option. It exploits the power of
MathJax~\cite{MathJax2.6} to translate {\LaTeX} expressions into MathML markup
and the Speech Rule Engine (SRE)~\cite{SRE} to translate the MathML into speech
strings that can be passed to a TTS.

The main focus of our work is to support students in all aspects of learning,
writing and working with {\LaTeX}. In particular, to allow them to take source
material, read it, browse it and manipulate it. We support learning {\LaTeX} by
making it easy to write and rearrange expressions and hear the effect, by using
MathJax's error reporting mechanism to indicate incorrect expressions and by the
possibility of reading expression aloud and interactively explore them on the
fly, while editing.  As editing mathematics is already a time consuming task we
aim to minimise the impact of having to listen to lengthy expressions multiple
times by exploiting Emacsspeak's feature of using rich prosody and SRE's ability
to summarise sub-expressions meaningfully for concise aural rendering of
expressions.


\section{Technical Realisation}
\label{sec:background}

Technically we realise our aims by combining Emacsspeak with MathJax and the
Speech Rule Engine.

\textbf{Emacsspeak} is a speech interface that allows visually impaired users to
interact independently and efficiently with the computer. Emacsspeak supports
audio formatting via W3C's Aural CSS (ACSS), which allows it to produce rich
aural presentations of electronic information.  For instance, syntax
highlighting is aurally rendered by changes in voice characteristic and
inflection. Combined with appropriate use of non-speech auditory icons this
creates the equivalent of spatial layout, fonts, and graphical icons important
in the visual interface. It provides rich contextual feedback and shifts some of
the burden of listening from the cognitive to the perceptual domain.  Emacsspeak
supports all activities in the Emacs editor, from implementing in a plethora of
programming languages, writing documents in markup formats like markdown, org or
{\LaTeX}, to browsing the file system and the WWW.

\textbf{Speech Rule Engine} (\sre) is a Javascript library that translates XML
expressions into speech strings according to rules that can be specified in an
extended Xpath syntax. It was originally designed for translation of MathML and
MathJax DOM elements for Google's ChromeVox screen reader~\cite{Sorge14}. It has
since been turned into a stand-alone application that can be installed server
side as a NodeJS application as well as used in client side web applications.
Besides the rules originally designed for the use in ChromeVox, it also has an
implemententation of the full set of Mathspeak rules. In addition it contains a
library for semantic interpretation and enrichment of MathML expressions, rules
for intelligent summarisation of sub-expressions, facilities for the interactive
exploration of mathematical expression, highlighting of DOM nodes, etc.  SRE
runs both in browsers as well as in NodeJS. It can be installed via Node's
package manager npm and offers a command line interface for batch translation of
XML expressions.


\textbf{MathJax} is a JavaScript library that ensures flawless rendering of
formulas across all platforms and browsers. It was originally built as a
polyfill solution for rendering mathematics in browsers until the MathML
standard would be sufficiently supported. But since the majority of browsers
still does not support MathML natively, MathJax has become the quasi standard
for displaying Mathematics on the web.  MathJax can render the most common
mathematical authoring formats ---- {\latex}, ASCIIMath, and MathML --- into
high quality output represented either as HTML or as SVG.  Although MathJax is
built as a client side solution, to be included into webpages, there is a
special NodeJS package release, which allows server side rendering of
mathematical expressions in one of the three input formats into SVG. In
addition, it offers a way to convert {\latex} into standardised MathML. MathJax
can also use \sre as a dependency to produce speech strings for mathematical
expression for inclusion into the rendered output.

\textbf{NodeJS Bridge:} The three systems are combined with a simple NodeJS
bridge, whereby MathJax and \sre run within an inferior JavaScript process in
Emacs. Emacsspeak communicates with both systems by message passing
between the different Emacs processes.  Using an inferior process over a simple
batch process integration does not only decrease potential lag due to delay in
system startup but also has the advantage that both integrated systems can keep
in independent internal state Emacsspeak can refer to.



\section{Reading {\LaTeX} in Emacsspeak}
\label{sec:rendering}

{\LaTeX} formulas are spoken in Emacsspeak in two different ways: Either
integrated into the continuous flow when the entire document is read or
speaking can be triggered interactively for a single formula. In either case
Emacsspeak detects the math delimiters and sends the expression to the node
process. MathJax then translates it into MathML and passes it to \sre, which in
turn performs a semantic analysis of the expression and then translates the
expression into a speech string using a chosen set of speech rules.

% \paragraph{Audio Formatting}

One of Emacsspeak's strength is that it enables audio formatting, by employing
changes in prosody (e.g., changes in pitch or stress), to indicate syntax
highlighting, special text elements like links etc. Similarly, \sre supports
prosody markup by allowing speech rules to define changes in pitch, rate, volume
and by adding pauses. We combine this by translating mathematical expressions in
\sre using a speech rule set that indicates two dimensional layout, such as sub
and superscripts or positions in fractions with changes in pitch and rate,
rather then employing more verbose indicator words like StartFraction or
EndFraction. \sre then returns translated formulas as sexpressions that contain
speech strings together with Aural CSS markup that Emacsspeak can exploit for
audio formatting.

% \paragraph{Interactive Exploration}

Since mathematical formulas are generally of a complex nature and just listening
to an expression once is generally not enough for a user to fully take in its
meaning.  It is particularly important to be able to engage with them
interactively by offering a means of traversing sub-expressions.  \sre provides
a number o different navigation options for mathematical expressions, which we
exploit from Emacsspeak. Thereby \sre keeps an internal state on the latest
mathematical formula that it has rendered. Emacsspeak then allows users to walk
this formula using simple cursor key navigation, while \sre produces speech
strings for the corresponding sub-expressions, potentially together with
positional information, such as denominator, numerator etc.



\section{Learning Support}
\label{sec:learning-features}

One major difficulty presented to students new to {\LaTeX} is learning the
intricacy of its syntax and in particular to correctly implement complex, nested
mathematical expressions.  We provide a number of features that are aimed in
particular at helping students understand expressions and write correct ones
themselves. For this we exploit as much as possible the abilities of the systems
we integrate:

\textbf{Error Handling} Since MathJax implements a version of the TeX
typesetting system it can return fairly detailed error reports. We expose these
directly to the reader together with an error earcon to allow users to quickly
understand and rectify problems in their input expressions.

\textbf{Syntax Highlighting} Since {\LaTeX} implements documents rather than
composes them in a WYSYG style, editors help users to distinguish content from
markup using syntax highlighting. Emacsspeak can exploit its audio formatting
achieving a similar effect, making it not only easy to separate content from
markup but also giving the reader an indication of what layout elements markup
would produce, such as bold type face, section headings etc. For math
expressions audio formatting is used to indicate nesting depth of braces, giving
a user an indication of the position within an expression they edit thus
highlighting potential problems or errors.


\textbf{Increased Editing Speed} Since editing {\LaTeX} can be a time consuming
task, which is only compounded by the need to listen to lengthy audio output of
mathematics, one of our main goals is to decrease the burden of listening.  One
way is by prefering rich audio formatting of more verbose tranditional reading
styles of mathematics (cf.~MathSpeak for example). A second way is by exploiting
the summarisation features \sre provides. For example, instead of reading out an
entire expression in detail \sre can use its semantic interpretation to produce
summaries such as ``sum with 5 summands'', which are only expanded when
explicitly prompted by a user. Emacsspeak uses this during editing of single
expressions. When the same expression is rendered multiple times, \sre can keep
an internal state of the expression allowing it to simply summarise
sub-expressions that have not been changed since the last editing state.


\bibliographystyle{plain}
\bibliography{assets16}

\end{document}


%%% Local Variables:
%%% mode: latex
%%% TeX-master: t
%%% End:
